\section{\textit{De-novo} assembly of sequenced reads}
\subsection{Installation of \denovo~assembler}
For \denovo~assembly we will use \texttt{mira} assembler.
You can download it from \url{http://sourceforge.net/projects/mira-assembler/}.
Click on Files $\rightarrow$ MIRA $\rightarrow$ stable. Rightclick on \texttt{mira\_4.0.2\_linux-gnu\_x86\_64\_static.tar.bz2}
and \texttt{Copy link address}.
%Click download. New page will open and copy link address of \textit{direct link}
To download it on our linux server, we will be using command \texttt{wget}. In linux terminal type: \\~\\
\texttt{cd \textasciitilde/\progDir} \\
\texttt{wget -O mira.tar.bz2} \\

and paste the copied location.
%wget -O mira.tar.bz2 http://downloads.sourceforge.net/project/mira-assembler/MIRA/stable/mira_4.0.2_linux-gnu_x86_64_static.tar.bz2?r=http%3A%2F%2Fsourceforge.net%2Fprojects%2Fmira-assembler%2F%3Fsource%3Dtyp_redirect&ts=1424879102&use_mirror=cznic
%We need to know the location of the program, and we
The program will be downloaded in our \texttt{\textasciitilde/\progDir} directory. 
The downloaded software is archived in .tar.bz2 format, therefore we need to extract it from archive.
To extract it from archive, type in terminal:\\~\\
\texttt{tar -jxvf mira.tar.bz2}\\

A new folder named \texttt{mira\_4.0.2\_linux-gnu\_x86\_64\_static} will appear.
This is our extracted software.
MIRA is already precompiled for us, so we just need
to find the compiled binary files in the folder and copy them to appropriate directory: \\~\\
\texttt{cd mira\_4.0.2\_linux-gnu\_x86\_64\_static} \\
\texttt{cd bin} \\
\texttt{cp mira \textasciitilde/\binDir} \\
\texttt{cd \textasciitilde} \\

%TODO: uztaisīt direktoriju denovo ngswrkdirā un tur iekopēt fastq failu un izveidot miras manifestfailu
Let's create a seperate directory for our \denovo~assembly project 
and copy the reads in it:\\~\\
\texttt{cd \workDir}\\
\texttt{mkdir \denovoDir}\\
\texttt{cd \denovoDir}\\
\texttt{cp \textasciitilde/\dataDir/\denovoReads.fastq~.}\\

MIRA needs a manifest file for performing \denovo~assembly.
We will create a basic manifest file. 
Our manifest file will consist of 5 entries. From MIRA's manual, these entries are:
%\begin{itemize}
%  \item \textbf{project =} tells the assembler the name you wish to give to the whole assembly project. MIRA will use that name throughout
%  the whole assembly for naming directories, files and a couple of other things.
%  You can name the assembly anyway you want, you should however restrain yourself and use only alphanumeric characters
%  and perhaps the characters plus, minus and underscore. Using slashes or backslashes here is a recipe for catastrophe.
%  \item \textbf{job =} tells the assembler what kind of data it should expect and how it should assemble it.
%  You need to make your choice mainly in three steps and in the end concatenate your choices to the [job=] entry of the
%  manifest:
%  \begin{enumerate}
%    \item 
%  \end{enumerate}
%\end{itemize}
\begin{itemize}
  \item \textbf{project} - name of our assemblies project. The project name will be used by MIRA in project's directory naming
  \item \textbf{job} -  tells the assembler whether
  \begin{enumerate}
    \item we want to perform \denovo~assembly or map reads against reference genome
    \item genomic DNA or transcripts were sequenced
    \item we want accurate (slow) or draft (fast) assembly
  \end{enumerate}
  \item \textbf{readgroup} - tells assembler which reads can be pooled together when assembling reads from multiple sequencing technologies
  \item \textbf{data} - tells the assembler where are our reads
  \item \textbf{technology} - tells the assembler what sequencing technology was used for generating reads
\end{itemize}
To create manifest file, we will use text editor \texttt{nano}. To start the text editor, type \\~\\
\texttt{nano}\\

and the editor will open. Now, to create the manifest file, in the text editor type:\\~\\
\texttt{project=\denovoReads\_Assembly\\
job=denovo,genome,draft\\
readgroup\\
data=\denovoReads.fastq\\
technology=iontor} \\

To save the text file hit \texttt{Ctrl o}, enter the name of the file (e.g. \texttt{\denovoReads.mnfst}),
hit \texttt{Enter} to save and \texttt{Ctrl x} to quit \texttt{nano}. To launch MIRA, type:\\~\\
\texttt{\textasciitilde/\binDir/mira \denovoReads.mnfst}\\

If you wish to gain finer control of some aspects of the assembling process,
then, please, do refer to the MIRA's manual.
