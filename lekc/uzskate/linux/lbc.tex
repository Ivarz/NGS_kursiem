\documentclass{article}
\usepackage[a4paper, total={6in, 8in}]{geometry}
\usepackage[usenames,dvipsnames]{xcolor}
\usepackage{framed}
\usepackage{fontspec}
\usepackage{hhline}
\usepackage{hyperref}
\usepackage{listings}
\newcommand{\git}{\texttt{git}}
\newcommand{\Git}{\texttt{Git}}
\newcommand{\IonTorrent}{\textit{IonTorrent}}
\newcommand{\binDir}{binaries}
\newcommand{\fbinDir}{\textasciitile/\binDir}
\newcommand{\progDir}{programs}
\newcommand{\workDir}{ngs\_work}
\newcommand{\denovoDir}{denovo}
\newcommand{\reseqDir}{reseq}
\newcommand{\refSeq}{chr\_merged.fa}
\newcommand{\refDir}{ref}
\newcommand{\refFullDir}{\texttt{\textasciitile/\workDir/\reseqDir/ref}}
\newcommand{\refRel}{\texttt{ref/\refSeq}}
\newcommand{\denovo}{\textit{de-novo}}
\newcommand{\denovoReads}{denovo\_reads}
\newcommand{\dataDir}{data/day1}
\newcommand{\mapReads}{reseq\_reads}
\newcommand{\snpEffData}{snpEffData}
\newcommand{\bash}{\textit{bash}}
\newcommand{\pipename}{snp\_pipeline.sh}
\newcommand{\targz}{\textit{.tar.gz}}
\newcommand{\tarbz}{\textit{.tar.bz2}}
\newcommand{\gz}{\textit{.gz}}
\begin{document}
\section{Basic linux commands}
Here some basic linux commands will be briefly explained.
%You don't need to memorize these, most frequently used commands w
For information about more commands, feel free to consult with the internet.

\subsection{Getting around}
\begin{tabular}{ll}
  Command & Explanation \\
  \hhline{==}
  \texttt{ls} & list contents of current directory \\
  \texttt{ls} \textit{dirname} & list contents of directory named \textit{dirname} \\
%  \textbf{cd} & \textbf{change directory} \\
  \texttt{cd} \textit{dirname} & change current directory to \textit{dirname} \\
  \texttt{cd \textasciitilde} & go to home directory (default directory) \\
  \texttt{cd ..} & go one directory up \\
  \texttt{pwd} & print name of current directory \\
  \texttt{mkdir} \textit{dirname} & make a new directory named \textit{dirname}\\
  \texttt{man} \textit{command} & show manual for \textit{command} \\
\end{tabular}

\subsection{File manipulation}
\begin{tabular}{ll}
  Command & Explanation \\
  \hhline{==}
%  cp & copy files and directories
%\textit{filename1} to \textit{filename2}
%  \textbf{cp} & \textbf{copy} \\
  \texttt{cp} \textit{filename1} \textit{filename2} & copy file \\
  \texttt{cp -r} \textit{dirname1} \textit{dirname2} & copy directory \\
  \texttt{mv} \textit{name1} \textit{name2} & move file or directory (can be used for renaming) \\
  \texttt{rm} \textit{filename} & remove file (you won't be able to recover removed file) \\
  \texttt{rm -r} \textit{dirname} & remove directory and its contents \\
  \texttt{more} \textit{filename} & command for paging through text file one screenful at a time \\%(\texttt{q} to quit)\\
  \texttt{less} \textit{filename} & similar as \texttt{more}. Better for viewing large text files (\texttt{q} to quit) \\
  \texttt{cat} \textit{filename1 filename2 filenameN} & concatenate text files in print output to terminal window \\
  \texttt{head} \textit{filename} & print first 10 lines of a text file to terminal window \\
  \texttt{head -n 20} \textit{filename} &  print first 20 lines of a text file to terminal window \\
  \texttt{tail} & opposite of \texttt{head} \\
  \texttt{wc} \textit{filename} & show the number of lines, words and characters in a text file \\
  \texttt{cut -f 2} \textit{tabDelimitedFilename} & extract 2nd column from a tab delimited file \\
  \texttt{cut -f 3 -d ,} \textit{comaSeperatedFilename} & extract 3rd column from a coma seperated file \\
  \texttt{nano} \textit{filename} & open \textit{filename} in text editor \texttt{nano} \\
\end{tabular}

\subsection{Archiving and unarchiving of files}
Note that for the \texttt{tar} utility, option \texttt{c} stands for compress, 
\texttt{x} - for uncompress or extract, \texttt{z} - for dealing with \textit{tar.gz},
and \texttt{j} - for dealing with \textit{tar.bz2}\\~\\
\subsubsection{Compressing}
\begin{tabular}{ll}
  Command & Explanation \\
  \hhline{==}
  \texttt{tar -cvf} \textit{filename.tar} \textit{filename} & \textbf{compress} file to \textbf{.tar} format\\
  \texttt{tar -zcvf} \textit{filename.tar.gz} \textit{filename} & \textbf{compress} file to \textbf{.tar.gz} format\\
  \texttt{tar -jcvf} \textit{filename.tar.bz2} \textit{filename} & \textbf{compress} file to \textbf{.tar.bz2} format\\
  \texttt{zip} \textit{filename.zip filename} & \textbf{compress} file to \textbf{.zip} format \\
  \texttt{gzip} \textit{filename} & \textbf{compress} file to \textbf{.gz} format \\
\end{tabular}

\subsubsection{Uncompressing}
\begin{tabular}{ll}
  Command & Explanation \\
  \hhline{==}
  \texttt{tar -xvf} \textit{filename.tar} & \textbf{uncompress} from \textbf{.tar} format \\
  \texttt{tar -zxvf} \textit{filename.tar.gz} & \textbf{uncompress} from \textbf{.tar.gz} format \\
  \texttt{tar -jxvf} \textit{filename.tar.bz2} & \textbf{uncompress} from \textbf{.tar.bz2} format \\
  \texttt{unzip} \textit{filename.zip} & \textbf{uncompress} from \textbf{.zip} format \\
  \texttt{gzip -d} \textit{file.gz} & \textbf{uncompress} from \textbf{.gz} format \\
\end{tabular}

\subsection{Input/Output redirection}
\begin{tabular}{ll}
  Command & Explanation \\
  \hhline{==}
  \textit{command} \texttt{>} \textit{filename} & output of \textit{command} is saved to \textit{filename}, overwriting it \\
  \textit{command} \texttt{>>} \textit{filename} & output of \textit{command} is appended at the end of \textit{filename} \\
  \textit{command} \texttt{<} \textit{filename} & \textit{command} reads input from \textit{filename} \\
%  \textit{command} \texttt{>} \textit{filename} & Whatever is printed out to terminal is saved to \textit{filename} instead, overwriting it \\
%  \textit{command} \texttt{>>} \textit{filename} & Whatever is printed out to terminal is appended at the end of \textit{filename} instead \\
  \textit{command1} \texttt{|} \textit{command2} & \textit{command2} takes the output of \textit{command1} and produces result \\  
%  \textit{command1} \texttt{|} \textit{command2} \texttt{>} \textit{filename} & \textit{command2} 
%    takes the output of \textit{command1} and saves result in \textit{filename} \\
%  \textit{command1} \texttt{&&} \textit{command2} & \textit{command2} is executed if \textit{command1} was executed successfully\\
%  \textit{command1} \texttt{||} \textit{command2} & \textit{command2} is executed if \textit{command1} has failed\\
\end{tabular}

\subsection{Lists of commands}
\begin{tabular}{ll}
  Command & Explanation \\
  \hhline{==}
  \textit{command1} \texttt{;} \textit{command2} & \textit{command2} is executed after \textit{command1}\\
  \textit{command1} \texttt{\&\&} \textit{command2} & \textit{command2} is executed if \textit{command1} was successful\\
  \textit{command1} \texttt{||} \textit{command2} & \textit{command2} is executed if \textit{command1} has failed\\
\end{tabular}

\subsection{Filters}
\begin{tabular}{ll}
  Command & Explanation \\
  \hhline{==}
  \texttt{grep} \textit{text filename} & Prints every line in \textit{filename} containing \textit{text} \\
  \texttt{sed 's/}\textit{red}\texttt{/}\textit{green}\texttt{/'} \textit{filename} & Prints every line in \textit{filename} 
     substituting word \textit{red} with word \textit{green} 
\end{tabular}

\subsection{Pattern matching}
%TODO:describe pattern usage. \\~\\
\begin{tabular}{cl}
  Pattern & Explanation \\
  \hhline{==}
  \texttt{*} & matches zero or more characters \\
  \texttt{?} & matches one character \\
  \texttt{\textasciitilde} & refers to user's default (home) directory
\end{tabular}

\subsection{Miscellaneous}
\begin{tabular}{ll}
  Command & Explanation \\
  \hhline{==}
  \texttt{echo} \textit{text} & display a line of text \\
  \texttt{history} & view your command line history \\
  \texttt{wget} \textit{someWebAddress} & download contents of \textit{someWebAddress} to current directory\\
  \texttt{make} & tool that is used to compile source code creating executables\\
  \texttt{export} \textit{name}\texttt{=}\textit{value} & sets \textit{value} to \textit{name}. Type \texttt{echo \$\textit{name}} to view \textit{value}\\
  \texttt{source} \textit{filename} & read and execute commands from the \textit{filename} argument \\
\end{tabular}  

\subsection{Syntax}
\begin{tabular}{cl}
  Syntax element & Explanation \\
  \hhline{==}
  \texttt{\textbackslash} & allows to split command over multiple lines\\
  
\end{tabular}
\subsection{\texttt{for} loop}
\texttt{for} loop allows to iterate over a list of items and apply commands on them:\\~\\
\texttt{for item in item1 item2 item3 item4; do\\
  \indent echo \$item \\
done}\\

Note that we are assigning to \texttt{item} each value in list of items 
(\texttt{item1 item2 item3 item4}) in the first line, and we are retrieving
the value of \texttt{item} by putting \$ before it.

We can also write \texttt{for} loop in a single line:\\~\\
\texttt{for item in item1 item2 item3 item4; do echo \$item; done}\\

\end{document}
