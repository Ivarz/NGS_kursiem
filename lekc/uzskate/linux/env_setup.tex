\section{Setup of the working environment}
%Let's create directories for our data:\\~\\
%\texttt{mkdir \progDir \\
%mkdir \workDir \\
%mkdir \binDir}\\~\\
Since we do not have the administrator's rights on this server, we can't install
software on the system. However, we can still install software locally in our home
directories. We will create a special directory where all our executables will be stored.
Let's create this directory and name it \texttt{\binDir}. In linux terminal type:\\~\\
\texttt{mkdir \binDir}\\

Let's make this directory special - every executable file we put there, we will be able
to easily execute, just by typing the executable's file name from anywhere in the system.
To achieve this, we will be adding \texttt{\textasciitilde/\binDir}~folder to the \texttt{\$PATH} 
system's environment variable by editing a text file named \texttt{.bashrc} using text editor \texttt{nano}.
In linux terminal type:\\~\\
\texttt{nano \textasciitilde/.bashrc}\\

Text editor will open the file and you can edit it. Navigate to the bottom of the text file using arrows
and type in following text:\\~\\
\texttt{export PATH=\textasciitilde/\binDir:\$PATH}\\

Hit \texttt{Ctrl o} and \texttt{Enter} to save file and \texttt{Ctrl x} to exit.
Reload \texttt{.bashrc}:\\~\\%TODO vēl kko pielikt klāt
\texttt{source .bashrc}\\

\begin{framed}
\texttt{\$PATH} variable contains a list of directories that the system will look in, when we are
entering a command. To view contents of \texttt{\$PATH}, type in terminal:\\~\\
\texttt{echo \$PATH}\\
\end{framed}
To install software we will simply have to copy executable to our special directory \texttt{\binDir}.
%The installation will be simply copying of executables to \texttt{\textasciitilde/\binDir}~
%wget https://github.com/samtools/samtools/archive/develop.zip

Many of the open source tools are deposited in the \url{https://github.com} repository.
To download software from \url{https://github.com} easily, we will use a tool called \git.
\git~is already preinstalled on our servers, however, on your own Ubuntu servers
you can install it by typing:\\~\\
\texttt{sudo apt-get install git}\\

%We will need a tool named samtools and we will download it with \git.
%To install samtools, type:\\~\\
%\texttt{cd \progDir}\\
%\texttt{git clone https://github.com/samtools/samtools}\\
%\texttt{git clone https://github.com/samtools/htslib}\\
%\texttt{cd samtools}\\
%\texttt{make}\\

%To test whether samtools compiled correctly, type:\\~\\
%\texttt{./samtools}\\

%If you see the program's interface, then the program was compiled correctly.
%Move compiled binary file to a directory where we will store our compiled software.\\~\\
%\texttt{cp samtools \textasciitilde/\binDir\\
%cd}\\

%Installation of \IonTorrent~mapping software \texttt{tmap}:\\~\\
%%https://github.com/iontorrent/TS/tree/master/Analysis/TMAP
%\texttt{cd \progDir}\\
%\texttt{git clone git://github.com/iontorrent/TMAP.git}\\
%\texttt{cd TMAP}\\
%\texttt{git submodule init}\\
%\texttt{git submodule update}\\
%\texttt{sh autogen.sh}\\
%\texttt{./configure}\\
%\texttt{make}\\

%Move the compiled binary file to our bin folder:\\~\\
%\texttt{cp tmap \textasciitilde/bin\\}
%cp ~/data/samtools. .

%(unzip)
%unzip samtools.zip

%cd samtools-develop

%(make)
%(zlib1g-dev)
%(libncurses5-dev)


