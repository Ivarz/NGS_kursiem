\subsection{SNP annotation and filtering with \texttt{snpEff} and \texttt{SnpSift}}
After generating a list of SNPs in \textit{VCF} format,
we will use a tool \texttt{snpEff} to predict whether they are causing
amino acid change and \texttt{SnpSift} to filter nonsynonymous and stop gained SNPs.

\subsubsection{Setting up and configuring \texttt{snpEff} and \texttt{SnpSift}}
\texttt{snpEff} and \texttt{SnpSift} are bundled together and are available at \url{http://snpeff.sourceforge.net}.
To download and install them, type in linux terminal:\\~\\
\texttt{cd \textasciitilde/\progDir} \\
\texttt{wget http://sourceforge.net/projects/snpeff/files/snpEff\_latest\_core.zip} \\

To \texttt{unzip snpEff} type:\\~\\
\texttt{unzip snpEff\_latest\_core.zip}\\

Unfortunately we wil not be able to launch both programs from \texttt{\binDir} folder
and we can leave the software in \texttt{\progDir} directory.

\texttt{SnpSift} and \texttt{snpEff} are \textit{.jar} files which means that
these are \texttt{java} applications. To run \texttt{java} applications through linux terminal,
use command \texttt{java -jar} \textit{filename.jar}. 
\texttt{snpEff} requires databases to predict effects of SNPs.
We will use human genome's (version \texttt{GRCh38.76}) database to annotate SNPs
and \texttt{snpEff}'s database \texttt{dbNSFP}.
The needed databases are already downloaded for us and resides in
\texttt{\textasciitilde/\dataDir/\snpEffData}. Our task is to 
configure \texttt{snpEff} to tell it where these databases can be found.

After downloading and unzipping of \texttt{snpEff} we will need to
edit \texttt{snpEff's} configuration file named \texttt{snpEff.config}
and change entry \texttt{data.dir = ./data/} to \texttt{data.dir = \textasciitilde/\dataDir/\snpEffData/}.
We will also add line \texttt{database.local.dbnsfp = \textasciitilde/\dataDir/snpEffData/dbNSFP2.7.txt.gz} to
tell \texttt{snpEff} to look for databases in these locations.

%In case you don't have these databases, 
%download them, using \texttt{snpEff}'s command \texttt{download} (it is quite large file (\textasciitilde 715~MB) and it may take some
%time to download it depending on the speed of internet connection)\\~\\
%In linux terminal type:\\~\\
%\texttt{cd \textasciitilde/\progDir/snpEff}\\
%\texttt{java -jar snpEff.jar download GRCh38.76}\\
%\texttt{nano snpEff.config}

%TODO aprakstīt abas komandas
\subsubsection{SNP annotation and filtering}
To sucessfully perfrom SNP
annotation, we need to provide \texttt{snpEff} with the command we are executing 
(\texttt{ann} - telling \texttt{snpEff} that we want to annotate SNPs), its configuration file (\texttt{-c snpEff.config}),
database we are annotating against (\texttt{GRCh38.76}) and our \textit{VCF} file. Since \texttt{snpEff} by default prints
output to terminal, we can save it using \texttt{>} operator.
To annotate called variants, type:\\~\\
\texttt{cd \textasciitilde/\workDir/\reseqDir}\\
\snpEff{\mapReads.bcftools\_snps.vcf}{\mapReads.bcftools\_snps.annotated.vcf}\\

\begin{framed}
If you see see an error saying \texttt{GC overhead limit exceeded} than you will have to give this
program more memory. To do this, type:\\~\\
\snpEffGC{\mapReads.bcftools\_snps.vcf}{\mapReads.bcftools\_snps.annotated.vcf}
\end{framed}
In the resulting \textit{VCF} file \texttt{snpEff} adds a new entry in the \texttt{INFO} field - \texttt{ANN}.
Explanation of this field can be found in \textit{VCF} file's header. See \url{http://snpeff.sourceforge.net/SnpEff_manual.html}
for more information about this tool and generated output files.

We can also annotate our \textit{VCF} file by using information from another \textit{VCF} file.
To do this, we will use \texttt{SnpSift}'s command \texttt{annotate}. 
We need to supply \texttt{SnpSift}'s \texttt{annotate} with the database \textit{VCF} file from which
the program will acquire information and \textit{VCF} file that we are going to annotate. We can
choose whether to annotate \texttt{ID}, \texttt{INFO} or both fields. In the example we will
annotate only \texttt{ID} fields by giving \texttt{-id} option to program.
To annotate called variants with \texttt{SnpSift}'s \texttt{annotate}, type:\\~\\
\SnpSiftAnnotate{\mapReads.bcftools\_snps.annotated.vcf}{\mapReads.bcftools\_snps.annotated.id.vcf}\\

\texttt{SnpSift} can also give information about SNP's: frequency in 1000 Genomes database,
\textit{SIFT}, \textit{Polyphen}, \textit{GERP} scores, etc. To perform this kind of annotation, we will use \texttt{SnpSift}'s command 
\texttt{dbnsfp}. To annotate with \texttt{SnpSift}'s \texttt{dbnsfp}, type:\\~\\
\SnpSiftAnnotateDBNSFP{\mapReads.bcftools\_snps.annotated.id.vcf}{\mapReads.bcftools\_snps.annotated.id.dbnsfp.vcf}\\

To filter only those SNPs that are marked as nonsynonymous or stop gained we are going to
use \texttt{SnpSift}'s command \texttt{filter}. \texttt{SnpSift}'s \texttt{filter} command needs
\textit{VCF} file annotated by \texttt{snpEff} and filtering expression.
We will use expression \texttt{"(ANN[*].EFFECT = 'missense\_variant)' || (ANN[*].EFFECT = 'stop\_gained')"}
which can be translated as: if any "ANN" field for this variant, has effect of missense or stop gain, then print this variant
to terminal. For more examples on filtering expressions, see \url{http://snpeff.sourceforge.net/SnpSift.html}.
To perform variant filtering, type:\\~\\
\SnpSiftFilter{\mapReads.bcftools\_snps.annotated.id.dbnsfp.vcf}{\mapReads.bcftools\_snps.annotated.id.dbnsfp.nonsyn\_stop.vcf}\\

A new \texttt{VCF} file named \texttt{\mapReads.bcftools\_snps.annotated.id.dbnsfp.nonsyn\_stop.vcf} will appear, in which 
there will be only missense and stop gain SNPs.

Now, suppose that we want to extract only chromosome, position, reference and alternate alleles, \textit{SIFT}
prediction and genotype from \textit{VCF} file. We can achieve it with \texttt{SnpSift}'s \texttt{extractFields} command,
which takes a \textit{VCF} file and a list of fields it needs extract. 
Type in terminal:\\~\\
\SnpSiftExtractFields{\mapReads.bcftools\_snps.annotated.id.dbnsfp.nonsyn\_stop.vcf}{\mapReads.bcftools\_snps.annotated.id.dbnsfp.nonsyn\_stop.txt}\\

A tab delimited file named \texttt{\mapReads.bcftools\_snps.annotated.id.dbnsfp.nonsyn\_stop.txt will appear}
which can be easily viewed in spreadsheet software.
%extractFields
%TODO vcf faila filtreshanas piemerus
