\section{Building variant calling pipeline}
We have \IonTorrent~targeted resequencing data from human chromosomes 1., 2. and 19. 
Our task is to find all nonsynonymous and stop mutations present in the data
and to automatize this process by building data analysis pipeline.
To accomplish this task we can divide our work in following subtasks:
\begin{itemize}
  \item Installation of relevant tools
  \item Obtaining of reference sequences
  \item Read mapping against reference genome
  \item Variant calling
  \item Variant annotation
  \item Filtering of nonsynonymous and stop mutation variants
  \item Pipeline building
\end{itemize}
\subsection{Tool installation}
\subsubsection{Installation of \IonTorrent~mapping software \texttt{tmap}}
To detect variants present in the data we need to map sequencing
reads against reference sequence. For reads generated with \IonTorrent~
sequencing platform we will use program \texttt{tmap}.
%Different short read mappers perform
%with various success depending on sequencing platform. 

\texttt{tmap} and its installation instructions can
be found at \url{https://github.com/iontorrent/TS/tree/master/Analysis/TMAP}.
Since we do not have the administrator's rights on this server, we can't install
software on the server. However, we can still use it locally.
Compilation instructions:\\~\\
%https://github.com/iontorrent/TS/tree/master/Analysis/TMAP
\texttt{cd \textasciitilde/\progDir}\\
\texttt{git clone git://github.com/iontorrent/TMAP.git}\\
\texttt{cd TMAP}\\
\texttt{git submodule init}\\
\texttt{git submodule update}\\
\texttt{sh autogen.sh}\\
\texttt{./configure}\\
\texttt{make}\\

Lets test the program to confirm that it was compiled successfully:\\~\\
\texttt{./tmap}\\

If you see the programs interface, the program was compiled successfully.
Move the compiled binary file to our \texttt{\binDir}~folder:\\~\\
\texttt{cp tmap \textasciitilde/\binDir}\\
%cp ~/data/samtools. .
%(unzip)
%unzip samtools.zip
%cd samtools-develop
%(make)
%(zlib1g-dev)
%(libncurses5-dev)
\subsubsection{Installation of \texttt{samtools} and \texttt{bcftools}}
We will also need two tools named \texttt{samtools} and \texttt{bcftools} which are used for
manipulation of mapped reads and variation calling. You can obtain these tools from
\url{http://sourceforge.net/projects/samtools/}. Click on \texttt{Files} $\rightarrow$ \texttt{samtools} $\rightarrow$ \texttt{1.2}
Rightclick on \texttt{samtools-1.2.tar.bz2} and choose \texttt{Copy link address}. 
We will download these tools in directory \texttt{\progDir}:\\~\\
\texttt{cd \textasciitilde/\progDir}\\
\texttt{wget -O samtools.tar.bz2}\\

paste the copied location and hit \texttt{Enter}.
Repeat this process for \texttt{bcftools}:\\~\\
\texttt{wget -O bcftools.tar.bz2}\\
paste the copied location and hit \texttt{Enter}.

The tools are compressed in \texttt{.tar.bz2}
format, so we need to extract them:\\~\\
\texttt{tar -jxvf samtools.tar.bz2}\\
\texttt{tar -jxvf bcftools.tar.bz2}\\

\begin{framed}
Note that if we had a lot more files to extract and it would be too
time consuming to manually extract them, we could use a \texttt{for}
loop and pattern matching to extract archives automatically:\\~\\
\texttt{for archive in *.tar.bz2; do}\\
\texttt{\indent tar -jxvf \$archive}\\
\texttt{done}\\
\end{framed}
%We will download its
%source code and library that it depends on with \git.
%To to download \texttt{samtools} and its library \texttt{htslib}, type:\\~\\
%\texttt{cd \progDir}\\
%\texttt{git clone https://github.com/samtools/samtools}\\
%\texttt{git clone https://github.com/samtools/htslib}\\

We have downloaded and extracted source code of the tools, but to make
these tools usable, we need to compile the source code. Source code compiling
is performed with command \texttt{make}:\\~\\
\texttt{cd samtools-1.2}\\
\texttt{make}\\
%\texttt{cd ../../bcftools-1.2}\\

If (hopefully) no errors were encountered, then \texttt{samtools} was compiled correctly. type:\\~\\
\texttt{./samtools}\\

to test the tool. If you see the program's interface, then the program was compiled successfully.
Move compiled binary file to a directory where we are storing our compiled software:\\~\\
\texttt{cp samtools \textasciitilde/\binDir\\}

Repeat the same process for \texttt{bcftools}
(go to \texttt{bcftools} source code directory,
compile it and copy resulting binary file to \texttt{\textasciitilde/\binDir})

\subsection{Obtaining reference sequences}
We will download reference sequences in a seperate directory to avoid file cluttering. Let's make
a new directory in our \texttt{\workDir}~directory named \texttt{\reseqDir}, and there we will create
a seperate folder \texttt{\refDir} for our reference sequences:\\~\\
\texttt{cd \textasciitilde}\\
\texttt{cd \workDir}\\
\texttt{mkdir \reseqDir}\\
\texttt{cd \reseqDir}\\
\texttt{mkdir \refDir}\\
\texttt{cd \refDir}\\

Our reference sequences can be accessed from a database made by
University of California, Santa Cruz.
The web address of the database is \url{http://genome.ucsc.edu/}.
To find the necessary references sequences for chromosomes 13., 17. and 20.
click on \texttt{Downloads} $\rightarrow$ \texttt{human} $\rightarrow$
\texttt{Data set by chromosome}. Right click on \texttt{chr13.fa.gz}
and choose \texttt{Copy link location}. In terminal type\\~\\
\texttt{wget} \\

paste the copied location and hit \texttt{Enter}. The reference sequence is compressed in \texttt{.gz}
format, so we need to extract it:\\~\\
\texttt{gzip -d chr13.fa.gz}\\

File \texttt{chr13.fa} will appear in our folder.

Repeat the process for chromosomes 17. and 20.

After we have obtained and extracted our reference sequences, we need to concatenate them.
To accomplish this task we will use command \texttt{cat}:\\~\\
\texttt{cat chr13.fa chr17.fa chr20.fa > \refSeq}\\

And we have the needed reference for further data processing steps.
%To enable read mapper to efficiently read and process reference sequence, we need to
%index reference sequence using mapping software's provided indexing function. Often generated
%indexes are incompatible between different read mappers, as a consequence, almost every read
%mapping software has their own indexing algorithms.

%To perform reference sequence indexing with \texttt{tmap} software, type in linux terminal:\\~\\


%\subsection{Obtaining reference sequences}
%We begin our task by obtaining reference genome. We will download reference
%sequences in a seperate directory to avoid file cluttering. Let's make
%a new directory in our \texttt{\workDir}~named \texttt{\reseqDir}, and there we will create
%a seperate folder \texttt{\refDir} for our reference sequences:\\~\\
%\texttt{cd \textasciitilde}\\
%\texttt{cd \workDir}\\
%\texttt{mkdir \reseqDir}\\
%\texttt{cd \reseqDir}\\
%\texttt{mkdir \refDir}\\
%\texttt{cd \refDir}\\

%Our reference sequences can be accessed from a database made by
%University of California, Santa Cruz.
%The web address of the database is \url{http://genome.ucsc.edu/}.
%To find the necessary references sequences for chromosomes 1., 2. and 19.
%click on \texttt{Downloads} $\rightarrow$ \texttt{human} $\rightarrow$
%\texttt{Data set by chromosome}. Right click on \texttt{chr1.fa.gz}
%and \texttt{Copy link location}.
%Downloads->Human
