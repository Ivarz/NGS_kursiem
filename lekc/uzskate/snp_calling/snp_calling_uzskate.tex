\section{Building variant calling pipeline}
We have \IonTorrent~targeted resequencing data from human chromosomes 1., 2. and 19. 
Our task is to find all nonsynonymous and stop mutations present in the data
and to automatize this process by building data analysis pipeline.
To accomplish this task we can divide our work in following subtasks:
\begin{itemize}
  \item Installation of relevant tools
  \item Obtaining of reference sequences
  \item Read mapping against reference genome
  \item Variant calling
  \item Variant annotation
  \item Filtering of nonsynonymous and stop mutation variants
  \item Pipeline building
\end{itemize}
\subsection{Obtaining reference sequences}
We will download reference sequences in a seperate directory to avoid file cluttering. Let's make
a new directory in our \texttt{\workDir}~directory named \texttt{\reseqDir}, and there we will create
a seperate folder \texttt{\refDir} for our reference sequences:\\~\\
\texttt{cd \textasciitilde}\\
\texttt{cd \workDir}\\
\texttt{mkdir \reseqDir}\\
\texttt{cd \reseqDir}\\
\texttt{mkdir \refDir}\\
\texttt{cd \refDir}\\

Our reference sequences can be accessed from a database made by
University of California, Santa Cruz.
The web address of the database is \url{http://genome.ucsc.edu/}.
To find the necessary references sequences for chromosomes 13., 17. and 20.
click on \texttt{Downloads} $\rightarrow$ \texttt{human} $\rightarrow$
\texttt{Data set by chromosome}. Right click on \texttt{chr13.fa.gz}
and choose \texttt{Copy link location}. In terminal type\\~\\
\texttt{wget} \\

paste the copied location and hit \texttt{Enter}. The reference sequence is compressed in \texttt{.gz}
format, so we need to extract it:\\~\\
\texttt{gzip -d chr13.fa.gz}\\

File \texttt{chr13.fa} will appear in our folder.

Repeat the process for chromosomes 17. and 20.

After we have obtained and extracted our reference sequences, we need to concatenate them.
To accomplish this task we will use command \texttt{cat}:\\~\\
\texttt{cat chr13.fa chr17.fa chr20.fa > \refSeq}\\

And we have the needed reference for further data processing steps.
%To enable read mapper to efficiently read and process reference sequence, we need to
%index reference sequence using mapping software's provided indexing function. Often generated
%indexes are incompatible between different read mappers, as a consequence, almost every read
%mapping software has their own indexing algorithms.

%To perform reference sequence indexing with \texttt{tmap} software, type in linux terminal:\\~\\


%\subsection{Obtaining reference sequences}
%We begin our task by obtaining reference genome. We will download reference
%sequences in a seperate directory to avoid file cluttering. Let's make
%a new directory in our \texttt{\workDir}~named \texttt{\reseqDir}, and there we will create
%a seperate folder \texttt{\refDir} for our reference sequences:\\~\\
%\texttt{cd \textasciitilde}\\
%\texttt{cd \workDir}\\
%\texttt{mkdir \reseqDir}\\
%\texttt{cd \reseqDir}\\
%\texttt{mkdir \refDir}\\
%\texttt{cd \refDir}\\

%Our reference sequences can be accessed from a database made by
%University of California, Santa Cruz.
%The web address of the database is \url{http://genome.ucsc.edu/}.
%To find the necessary references sequences for chromosomes 1., 2. and 19.
%click on \texttt{Downloads} $\rightarrow$ \texttt{human} $\rightarrow$
%\texttt{Data set by chromosome}. Right click on \texttt{chr1.fa.gz}
%and \texttt{Copy link location}.
%Downloads->Human
