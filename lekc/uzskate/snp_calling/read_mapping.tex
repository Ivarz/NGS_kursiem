\subsection{Read mapping against reference sequence}
%TODO fastq
%The sequenced reads we obtain from sequencing platform usually are in \textit{FASTQ} format.
%Each record in \textit{FASTQ} file consists of four entries:
%\begin{enumerate}
%  \item read ID, beginning with symbol @
%  \item DNA sequence of read
%  \item symbol +
%  \item \textit{ASCII} encoded quality of each nucleotide in read
%\end{enumerate}

To enable read mapper to efficiently read and process reference sequence, we need to
index reference sequence using mapping software's provided indexing function. Often generated
indexes are incompatible between different read mappers. As a consequence, almost every read
mapping software has their own indexing algorithms.

To perform reference sequence indexing with \texttt{tmap} software, type in linux terminal:\\~\\
\texttt{\refindex{\refRel}}\\

%TODO par map1-2-3 pastastit un pateikt rtfm
To perform the actual read mapping against our reference, type:\\~\\
\tmap{\refRel}{\mapReads.fastq}{\mapReads\_mapped.bam}\\

We have obtained \textit{BAM} file. \textit{BAM} stands for \textbf{B}inary S\textbf{AM} file.
\textit{SAM}, in turn, stands for \textbf{S}equence \textbf{A}lignment/\textbf{M}ap format.

\subsubsection{\textit{BAM} file viewing and manipulation}
%Lekcija par bam failu
Let's find out what is in the first 20 lines of \textit{BAM} file using \texttt{samtools} and \texttt{head}:\\~\\
\texttt{samtools view \mapReads\_mapped.bam | head -n 20}\\

%TODO aprakstīt katru no soļiem
To view \textit{BAM} file header, use 
\texttt{-H} option in \texttt{samtools view} command:\\~\\
\texttt{samtools view -H \mapReads\_mapped.bam}\\

As you can see, we have not defined our read group identifier (\texttt{ID:NOID})
and sample name (\texttt{SM:NOSM}). We can correct it using \texttt{samtools}
command \texttt{reheader}. It takes a \textit{BAM} header and \textit{BAM} file
as input. It replaces \textit{BAM} file's header with the provided header and 
outputs new \textit{BAM} file.

We will read our \textit{BAM} files header and modify using \texttt{sed}
utility. The modified header will be given to \texttt{samtools reheader}
along with the original bam file. The output will be saved using \texttt{>}
operator:\\~\\
\reheader{\mapReads\_mapped.bam}{\mapReads\_mapped.reheaded.bam}\\

Alternatively, we could have simply defined the read group ID and sample name during mapping process,
using option \texttt{-R}:\\~\\% with the complete tab delimited \texttt{RG} line of bam header:\\~\\
\tmapWithRG{\refRel}{\mapReads.fastq}{\mapReads\_mapped.reheaded.bam}\\

%\begin{framed}
%Note that the tab is denoted as \texttt{\textbackslash t} in our header line.
%\end{framed}

Detailed information about \textit{SAM/BAM} file format, see \url{http://samtools.github.io/hts-specs/SAMv1.pdf}.\\~\\

To view alignment in a more human readable format we will need to sort and index
our \textit{BAM} file. To do this will use \texttt{samtools} commands 
\texttt{sort} and \texttt{index}:\\~\\

\texttt{samtools sorted \mapReads\_mapped.bam \mapReads\_mapped.sorted}\\
\texttt{samtools index \mapReads\_mapped.sorted.bam}\\
%tview chr13:27925825
%region
%idxstats
