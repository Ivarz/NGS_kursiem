\subsection{Pipeline building}
%\refindex{\refRel}
%\tmap{\refRel}{\mapReads.fastq}{\mapReads\_mapped.bam}\\~\\
Writing all these commands by hand is time consuming, tedious and error prone
and we need a better way to execute these commands.
One way how to solve this problem is to save all our commands
in a text file and, when the time comes to analyze our data, copy these
commands in the terminal. While this method can save our time on looking
in manuals for the correct commands, it is still time consuming to wait
for each command to end, so that we can copy the next one in. 
We need a better way how to automatically execute these commands and
here \bash~ scripting comes in handy.

\bash~ stands for \textit{\textbf{B}ourne-\textbf{A}gain \textbf{Sh}ell}
and is one of the most popular choices for perfoming shell scripting. 
Whenever we typed a command in linux terminal it was shell that was
performing all the commands and giving us output.
So we have been using \bash~all along not knowing it.

Starting a \bash~script is really easy - open a text editor, on the
first line write \texttt{\#!/bin/bash}, on the next lines some commands,
and save the file. You can tell \bash~to launch script by typing in terminal:\\~\\
\texttt{bash }\textit{ yourFileName}\\

To build SNP calling pipeline we will use exactly this approach -
open text editor and copy following lines (note that text starting
with \texttt{\#} is a comment line where you can write anything you want to make
the code readable):\\~\\
\lstinputlisting{snp_calling/pipelines/pipeline1.sh}
%\begin{listing}
%\texttt{\#!/bin/bash}\\~\\
%\tmapWithRG{\refRel}{\mapReads.fastq}{\mapReads\_mapped.reheaded.bam}\\~\\
%\texttt{samtools sort \mapReads\_mapped.reheaded.bam \mapReads\_mapped.reheaded.sorted}\\~\\
%\reheader{\mapReads\_mapped.bam}{\mapReads\_mapped.reheaded.bam}\\~\\
%\texttt{\samtoolssnp{\refRel}{\mapReads\_mapped.reheaded.sorted.bam}{\mapReads.bcftools\_snps.vcf}}\\~\\
%\snpEff{\mapReads.bcftools\_snps.vcf}{\mapReads.bcftools\_snps.annotated.vcf}\\~\\
%\SnpSiftFilter{\mapReads.bcftools\_snps.annotated.vcf}{\mapReads.bcftools\_snps.annotated.nonsyn\_stop.vcf}\\~\\
%\end{lstlisting}
Save it as \texttt{\pipename} and launch it by typing in terminal:\\~\\
\texttt{bash \pipename}\\

This is good if we want to repeat this analysis over and over again on this one sample.
What if we have a new sample we want to analyse? We could replace filenames of previous sample,
to filenames of the new sample by hand. However, there is a better way - we can save our
sample name in a variable, and use that variable throughout the script. In \bash~
we assign value to a variable with \texttt{=} operator, and retrieve value from variable
by adding \texttt{\$} at the beginning of variable's name:
\lstinputlisting{snp_calling/pipelines/pipeline2.1.sh}
%\texttt{\#!/bin/bash}\\~\\
%\texttt{sample\_name=\mapReads}\\~\\
%\tmapWithRG{\refRel}{\$sample\_name.fastq}{\$sample\_name.mapped.reheaded.bam}\\~\\
%\texttt{samtools sort \$sample\_name.mapped.reheaded.bam \$sample\_name.mapped.reheaded.sorted}\\~\\
%%\reheader{\mapReads\_mapped.bam}{\mapReads\_mapped.reheaded.bam}\\~\\
%\texttt{\samtoolssnp{\refRel}{\$sample\_name.mapped.reheaded.sorted.bam}{\$sample\_name.bcftools\_snps.vcf}}\\~\\
%\snpEff{\$sample\_name.bcftools\_snps.vcf}{\$sample\_name.bcftools\_snps.annotated.vcf}\\~\\
%\SnpSiftFilter{\$sample\_name.bcftools\_snps.annotated.vcf}{\$sample\_name.bcftools\_snps.annotated.nonsyn\_stop.vcf}\\~\\
%We can add more flexibility in our script

We can go even further - we can provide sample name as an argument
to our pipeline. There is a predefined \bash~ variable (\texttt{\$1})
that stores everything we supply as an argument for our script:
\lstinputlisting{snp_calling/pipelines/pipeline3.sh}

To launch it, type:\\~\\
\texttt{bash \pipename~\mapReads}\\

Suppose that we have multiple \textit{FASTQ} files that we want to analyze.
Using pattern matching and \texttt{for} loop, we can automatically process them with our pipeline:\\~\\
\texttt{for fastqFile in *.fastq; do \\
\indent bash \pipename~\$fastqFile \\
done
}

