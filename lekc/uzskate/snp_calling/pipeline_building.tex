\subsection{Pipeline building}
%\refindex{\refRel}
%\tmap{\refRel}{\mapReads.fastq}{\mapReads\_mapped.bam}\\~\\
Writing all these commands by hand is time consuming, tedious and error prone
and we need a better way how execute these commands.
One way how to solve this problem is to save all our commands
in a text file and, when the time comes to analyze our data, copy these
commands in the terminal. While this method can save our time on looking
in manuals for the correct commands, it is still time consuming to wait
for each command to end, so that we can copy the next one in. 
We need a better way how to automatically execute these commands and
here \bash~ scripting comes in handy.

\bash~ stands for \textit{\textbf{B}ourne-\textbf{A}gain \textbf{Sh}ell}
and is one of the most popular choices for perfoming shell scripting. 
Whenever we typed a command in linux terminal it was shell that was
performing all the commands and giving as output.
So we have been using \bash~ all along not knowing it.

Starting a \bash~ script is really easy - open a text editor, on the
first line write \texttt{\#!/bin/bash}, on the next lines some commands,
and save the file. You can tell \bash~ to launch script by typing in terminal:\\~\\
\texttt{bash }\textit{ yourFileName}\\

To build SNP calling pipeline we will use exactly this approach -
open text editor and copy following lines:\\~\\
\texttt{\#!/bin/bash}\\
\tmapWithRG{\refRel}{\mapReads.fastq}{\mapReads\_mapped.reheaded.bam}\\~\\
%\reheader{\mapReads\_mapped.bam}{\mapReads\_mapped.reheaded.bam}\\~\\
\texttt{\samtoolssnp{\refRel}{\mapReads\_mapped.reheaded.sorted.bam}{\mapReads.bcftools\_snps.vcf}}\\~\\
\snpEff{\mapReads.bcftools\_snps.vcf}{\mapReads.bcftools\_snps.annotated.vcf}\\~\\
\SnpSiftFilter{\mapReads.bcftools\_snps.annotated.vcf}{\mapReads.bcftools\_snps.annotated.nonsyn\_stop.vcf}\\~\\

Save it as \texttt{\pipename} and launch it by typing in terminal:\\~\\
\texttt{bash \pipename}
