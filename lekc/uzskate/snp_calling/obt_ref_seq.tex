\subsection{Obtaining reference sequences}
We will download reference sequences in a seperate directory to avoid file cluttering. Let's make
a new directory in our \texttt{\workDir}~directory named \texttt{\reseqDir}, and there we will create
a seperate folder \texttt{\refDir} for our reference sequences:\\~\\
\texttt{cd \textasciitilde}\\
\texttt{cd \workDir}\\
\texttt{mkdir \reseqDir}\\
\texttt{cd \reseqDir}\\
\texttt{mkdir \refDir}\\
\texttt{cd \refDir}\\

Our reference sequences can be accessed from a database made by
University of California, Santa Cruz.
The web address of the database is \url{http://genome.ucsc.edu/}.
To find the necessary references sequences for chromosomes 1., 2. and 19.
click on \texttt{Downloads} $\rightarrow$ \texttt{human} $\rightarrow$
\texttt{Data set by chromosome}. Right click on \texttt{chr1.fa.gz}
and choose \texttt{Copy link location}. In terminal type\\~\\
\texttt{wget} \\

paste the copied location and hit \texttt{Enter}. The reference sequence is compressed in \texttt{.gz}
format, so we need to extract it:\\~\\
\texttt{gzip -d chr1.fa.gz}\\

File \texttt{chr1.fa} will appear in our folder.

Repeat the process for chromosomes 2. and 19.

After we have obtained and extracted our reference sequences, we need to concatenate them.
To accomplish this task we will use command \texttt{cat}:\\~\\
\texttt{cat chr1.fa chr2.fa chr19.fa > \refSeq}\\


